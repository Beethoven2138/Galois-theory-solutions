\documentclass{article}
\usepackage{graphicx} % Required for inserting images
\usepackage[english]{babel}
\usepackage{amssymb}
\usepackage{amsthm}
\usepackage{enumitem} 
\usepackage{amsmath}
\usepackage{amsfonts}
\usepackage{tikz}
\usetikzlibrary{matrix}
\usepackage[english]{babel}
\usepackage{mathtools}
\usepackage[a4paper, total={6in, 8in}]{geometry}
\usepackage{cite}
\title{Galois Theory Solutions}
\author{Saxon Supple}
\date{July 2024}
\usepackage{centernot}

\newtheorem{theorem}{Theorem}[section]
\newtheorem{definition}[theorem]{Definition}
\newtheorem{lemma}[theorem]{Lemma}
\newtheorem{proposition}[theorem]{Proposition}
\newtheorem{corollary}[theorem]{Corollary}
\newtheorem{example}[theorem]{Example}
\newtheorem{remark}[theorem]{Remark}
\newtheorem{exercise}[theorem]{Exercise}

\begin{document}

\maketitle

\section{Groups}

\subsection{Groups}
\begin{exercise}
Suppose That $G$ is a group. Show that the identity element $e$ is unique, and that for each $g\in G$ the inverse element $g^{-1}$ is also unique.
\end{exercise}
\begin{proof}
Let $e_1$ and $e_2$ be identity elements of $G$. Then $e_1=e_1e_2=e_2$ so identity elements are unique.\\
Let $g_1^{-1}$ and $g_2^{-1}$ be inverses of $g$. Then $e=g_1^{-1}g=g_2^{-1}g\implies g_1^{-1}=g_2^{-1}$ so inverses are unique.
\end{proof}

\begin{exercise}
Show that if $H$ is a subgroup of $\mathbb{Z}$, then $H$ is cyclic.
\end{exercise}
\begin{proof}
There exists a greatest common divisor of the elements of $H$. Call it $d$. $d\in H$ by Bezout's lemma. Furthermore, every element of $H$ is a multiple of $d$. Thus $d$ is a generator of $H$ so $H$ is cyclic.
\end{proof}

\begin{exercise}
Show that a subgroup $F$ of a cyclic group $G$ is cyclic.
\end{exercise}
\begin{proof}
Let $g$ be a generator of $G$. Then every element $f\in F$ is of the form $g^n$ for some $n\in \mathbb{Z}$. Let $H$ be the set of all integers $n$ such that $g^n\in F$. $e=g^0\in F$ so $0\in H$. $g^n\in F\implies g^{-n}\in F$ so $H$ is closed under inverses. $g^n,g^m\in F\implies g^ng^m=g^{n+m}\in F$ so $H$ is closed under composition. Thus $H$ is a subgroup of $\mathbb{Z}$ so is cyclic. Let $d$ be a generator of $H$. every element of $F$ is then of the form ${(g^d)}^k$ for some integer $k$. Thus $F$ is cyclic.
\end{proof}

\begin{exercise}
Suppose that $A$, $A_1$ and $A_2$ are subsets of a group $G$, and that $A_1\subseteq A_2$.
Show that\\
\begin{align*}
(i&) Z(A)\text{ is a subgroup of }G\\
(ii&) Z(A_2)\subseteq Z(A_1)\\
(iii&) A\subseteq Z(Z(A))\\
(iv&) Z(A)=Z(Z(Z(A)))\\
(v&) A\subseteq Z(A)\text{ if and only if }A\text{ is abelian.}
\end{align*}
\end{exercise}

\begin{proof}
$(i)$\\
$e^{-1}g^{-1}eg=e\forall g\in G$ so $e\in Z(A)$.\\
Let $a\in Z(A)$. Let $g\in A$. Then 
\begin{align*}   
a^{-1}g^{-1}ag&=e\\\implies ag&=ga\\\implies g^{-1}ag&=a\\\implies g^{-1}a&=ag^{-1}\\\implies ag^{-1}a^{-1}&=g^{-1}\\\implies ag^{-1}a^{-1}g&=e
\end{align*}
so $a^{-1}\in Z(A)$.

Let $a,b\in Z(A),g\in A.$ $(ab)^{-1}g^{-1}(ab)g=b^{-1}a^{-1}g^{-1}abg=b^{-1}g^{-1}bg=e$ so $ab\in Z(A)$.
Thus $Z(A)$ is a subgroup of $G$.

$(ii)$\\
Let $a\in Z(A_2)$. Let $g\in A_1$. Then $g\in A_2$ so $[a,g]=e$. Thus $a\in Z(A_1)$.

$(iii)$\\
Let $a\in A$. We want $[a,g]=e\forall g\in Z(A)$. If $g\in Z(A)$, then $[g,a]=e\implies [a,g]=e$.

$(iv)$\\
$A\subseteq Z(Z(A))$ so $Z(Z(Z(A)))\subseteq Z(A)$. Furthermore, $Z(A)\subseteq Z(Z(Z(A)))$ so $Z(A)=Z(Z(Z(A)))$.

$(v)$\\
Let $a,b\in A$. Then $a\in Z(A)$ so $[a,b]=0$ so $A$ is abelian.

Now let $A$ be abelian. Then every element of $A$ commutes with every element of $A$ so $A\subseteq Z(A)$.
\end{proof}

\begin{exercise}
Show that if $\phi$ is a homomorphism of a group $G$ into a group $H$ then its kernel is a normal subgroup of $G$.
\end{exercise}
\begin{proof}
let $a\in\text{ker }\phi$ so that $\phi(a)=e$. Let $g\in G$. Then $\phi(g^{-1}ag)=\phi(g^{-1})\phi(a)\phi(g)=\phi(g)^{-1}\phi(g)=e$ so $a^g\in \text{ker }\phi$. Thus $\text{ker }\phi$ is normal.
\end{proof}

\begin{exercise}
Suppose that $A$ is a non-empty subset of a group $G$ and that $g\in G$. let $\psi_A(g)=A^g$. Show that $\psi_A$ maps $G$ onto conj$(A)$, and $\psi_A(g')=\psi_A(g)$ if and only if $g'\in N(A)g$. Thus if $G$ is a finite group then $|\text{conj}(A)|=|G/N(A)|$, so that $|\text{conj}(A)|\cdot|N(A)|=|G|$.
\end{exercise}
\begin{proof}
$\psi_A(g)=A^g$ so $\psi_A(g)$ is conjugate to $A$. $\psi_A$ is also clearly surjective so maps $G$ onto conj$(A)$.

$\psi_A(g')=\psi_A(g)\iff A^{g'}=A^g$.

Let $g'\in N(A)g$. Then $g'=hg$ for some $h\in G$ such that $a^h\in A\forall a\in A$. Thus given $a\in A$, $a^{g'}=(hg)^{-1}ahg=g^{-1}h^{-1}ahg=g^{-1}a^hg\in A^g$. Hence $A^{g'}\subseteq A^g$. Furthermore, $g=h^{-1}g'$ so $A^g\subseteq A^{g'}$. Thus $A^{g'}=A^g$.

Now let $A^{g'}=A^g$. We want $a^{g'g^{-1}}\in A\forall A$. Let $a\in A$. $a^{g'g^{-1}}=gg'^{-1}ag'g^{-1}=ga^{g'}g^{-1}=(b^g)^{g^{-1}}$ for some $b\in A$. Thus $a^{g'g^{-1}}\in A$.

Let $G$ be finite. We have established a correspondence between elements of conj$(A)$ and cosets of $N(A)$ so $|\text{conj}(A)|=|G/N(A)|$. $|\text{conj}(A)|\cdot|N(A)|=|G|$ since $|G/N(A)|=|G|/|N(A)|$. 
\end{proof}

\begin{exercise}
Show that if $G$ is a finite group and $G$ is a subgroup of $G$ with index $2$ in $G$, then $H$ is a normal subgroup of $G$.
\end{exercise}
\begin{proof}
Let $H$ and $gH$ be the cosets in $G/H$ with $g\notin H$. Let $a\in G$. Then $aH=H$ or $aH=gH$. If $aH=H$, then $a\in H$ so $aH=H=Ha$. If $aH=gH$, then $a\notin H$ so $Ha\neq H$ so $Ha=gH=aH$. Thus the left and right cosets of $H$ are the same so $H$ is normal.
\end{proof}

\begin{exercise}
Show that if $G$ is a group, then $G/[G,G]$ is abelian, and if $H\trianglelefteq G$ then $G/H$ is abelian if and only if $[G,G]\subseteq H$.
\end{exercise}
\begin{proof}
Let $a,b\in G$. $(a[G,G])^{-1}(b[G,G])^{-1}a[G,G]b[G,G]=[a,b][G,G]=e[G,G]$ so $G/[G,G]$ is abelian.

$(\implies)$ Let $a,b\in G$. $[a,b]H=[aH,bH]=H$ so $[a,b]\in H$. Thus $[G,G]\subseteq H$.

$(\impliedby)$ Let $aH,bH\in G/H$. $[aH,bH]=[a,b]H=eH$. Thus $G/H$ is abelian.
\end{proof}

\begin{exercise}
Suppose that $G$ has exactly one subgroup $H$ of order $k$. Show that $H$ is a normal subgroup of $G$.
\end{exercise}
\begin{proof}
Let $g\in G$. $|H^g|=H$ so $H^g=H$. Thus $H$ is normal.
\end{proof}


\begin{exercise}
Suppose that $H$ is a normal subgroup of $G$ and that $K$ is a normal subgroup of $H$. Is $K$ necessarily a normal subgroup of $G$?
\end{exercise}
\begin{proof}
No. Let $G=S_4$, $H=<(12)(34)>$, $K=\{(12)(34),(13)(42),(23)(41),e\}$.
\end{proof}


\begin{exercise}
Show that a group $G$ is generated by each of its elements (other than the identity element) if and only if $G$ is a finite cyclic group of prime order.
\end{exercise}
\begin{proof}
$(\impliedby)$ Let $g$ be a generator of $G$ and let $G$ have order $p$ for $p$ a prime. Let $g^n$ be any element of $G$ that isn't the identity. Then $n$ is not a multiple of $p$ and so is coprime to $p$. Thus for any $a\in \mathbb{Z}$ there exists an $x\in \mathbb{Z}$ such that $nx\equiv a$ mod $p$ implying $(g^n)^x=g^a$. Thus $g^n$ is a generator.

$(\implies)$ $G$ is cyclic by definition. Suppose that $G$ is infinite. Let $g\in G$ be a generator. Then $g^2$ is also a generator so there exists an $n\in \mathbb{Z}$ such that $g^{2n}=g\implies g^{2n-1}=e$. But then $g$ has finite order; a contradiction. Thus $G$ is finite. Now suppose that $G$ is of composite order. Write $|G|$ as $ab$ where $a$ and $b$ are positive integers greater than $1$. Let $g$ be a generator of $G$. Then $g^a$ is also a generator of $G$. However, $g^a$ has order $b<|G|$ so is not a generator; a contradiction. Thus $|G|$ is prime.
\end{proof}

\begin{exercise}
Describe the elements of $\mathbb{Z}_n$ which generate $\mathbb{Z}_n$.
\end{exercise}
\begin{proof}
The numbers coprime to $n$.
\end{proof}


\begin{exercise}
Give an example of a non-abelian group of order $8$ all of whose subgroups are normal.
\end{exercise}
\begin{proof}
$Q_8$.
\end{proof}

\subsection{Finite Abelian Groups}

\begin{exercise}
Suppose that $G$ is a finite abelian group for which every element other than the identity has order $k$. Show that $k$ is a prime number, and that $G$ is isomorphic to the product of cyclic groups, each of order $k$.
\end{exercise}
\begin{proof}
Suppose that $k$ is composite. Let $k=ab$ where $a,b>1$. Let $g\in G$ not be the identity so have order $k$. Then $g^a\neq 0$ has order $b$; a contradiction. Thus $k$ is prime.

We have that $G\cong \mathbb{Z}_{d_1}\oplus...\oplus \mathbb{Z}_{d_n}$, where each $d_i$ is a prime power. Let $\phi:\mathbb{Z}_{d_1}\oplus...\oplus \mathbb{Z}_{d_n}\to G$ be the isomorphism. Let $a_i=(0,...,0,1,0,...,0)$ ie, $1$ in the $i$th place and $0$ elsewhere. Then $d_i=o(a_i)=o(\phi(a_i))=k$. Thus $G\cong \mathbb{Z}_k\oplus...\oplus \mathbb{Z}_k$.
\end{proof}

\begin{exercise}
Suppose that $a$ and $b$ are positive integers with highest common factor $d$. Show that $\mathbb{Z}_a\oplus \mathbb{Z}_b\cong\mathbb{Z}_d\oplus\mathbb{Z}_{ab/d}$.
\end{exercise}
\begin{proof}
\end{proof}

\begin{exercise}
Suppose that $G$ is an abelian group. Show that the set $T$ of elements of finite order is a subgroup of $G$ and that every element of $G/T$, except the identity, is of infinite order.
\end{exercise}
\begin{proof}
$e\in T$. Let $a,b\in T$ with orders $n$ and $m$ respectively. Then $(ab)^{nm}=(a^n)^m(b^m)^n=e$ so $ab\in T$. If $a\in T$ has order $n$ then $a^{-1}$ has order $n$. Thus $T$ is a subgroup of $G$. Let $gT\in G/T$ where $g\notin T$. Then $g$ has infinite order. Suppose there exists an integer $n$ such that $(gT)^n=T$. Then $g^n$ has finite order so $g$ has finite order; a contradiction.
\end{proof}

\begin{exercise}
Suppose that $G$ is a finitely generated abelian group every element of which, except the identity, has infinite order. Show that $G\cong\mathbb{Z}^s$, where $s$ is defined by the property that $G$ is generated by $s$ elements, but is not generated by $s-1$ elements.
\end{exercise}
\begin{proof}
Let $g_1,...,g_s$ be a generating set of $G$. Define $\phi:\mathbb{Z}^s\to G:(a_1,...,a_s)\mapsto a_1g_1+...+a_sg_s$. $\phi$ is a homomorphism and surjective. Suppose that $\phi$ is not injective. Let $m$ be the smallest positive integer such that there exists a set of generators $\{h_1,...,h_s\}$ and a relation $mh_1+a_2h_2+...+a_sh_s=0$. $m>1$ since otherwise $G$ would be generated by $h_2,...,h_s$. We can write $a_i=mq_i+r_i$ where $0\leq r_i<m$ for $2\leq i \leq s$. Then if $w=h_1+q_2h_2+...+q_sh_s$, $G$ is generated by $\{w,h_2,...,h_s\}$ and $mw_1+r_2h_2+...+r_sh_s=mh_1+(a_2-r_2)h_2+...+(a_s-r_s)h_s+r_2h_2+...+r_sh_s=mh_1+a_2h_2+...+a_sh_s=0$. The minimality of $m$ implies that $r_2=...=r_s=0$ and so $mw=0$. But then $w$ has finite order while not being the identity (since $m>1$ so $h_1+q_2h_2+...+q_sh_s\neq0$); a contradiction. Thus $\phi$ is injective so an isomorphism.
\end{proof}

\begin{exercise}
Suppose that $G$ is a finitely generated abelian group. Show that $G\cong \mathbb{Z}^s\oplus T$, where $T$ is a finite group.
\end{exercise}
\begin{proof}
Let $\{g_1,...,g_n\}$ be a generating set of $G$. Let $F$ be generated by all $g_i$ of infinite order and let $T$ be generated by all $g_i$ of finite order so that $G\cong F\oplus T$. The result then follows by the above exercise.
\end{proof}

\subsection{Finite Permutation Groups}
\begin{exercise}
By considering conjugacy classes, show directly that $A_5$ is simple.
\end{exercise}
\begin{proof}
$|A_5|=60$ so a subgroup of $A_5$ will have order $1,2,3,4,5,6,10,12,15,20,30$ or $60$. A normal subgroup is a union of conjugacy classes. The conjugacy classes of $A_5$ have $1,12,12,15$ and $20$ elements. The only possibility for unions of those conjugacy classes which include the trivial group are the trivial group and $A_5$ so $A_5$ is simple. 
\end{proof}

\begin{exercise}
Show that the group of rotations of the cube has $24$ elements. By considering its four diagonals, show that it is isomorphic to $S_4$. By considering the three pairs of opposite faces, show that there is an epimorphism of $S_4$ onto $S_3$.
\end{exercise}
\begin{proof}
A rotation of a cube is uniquely characterized by the final position and rotation of one of its faces. A face can be moved to $6$ different places and can then be placed in $4$ different rotations so the group of rotations of the cube has $24$ elements.

Each permutation of the four diagonals of the cube can be attained by a rotation of the cube and the only rotation which yields the trivial permutation of the diagonals is the trivial rotation so by the first isomorphism theorem the two groups are isomorphic. The group of permutations of the diagonals is then isomorphic to $S_4$ so the group of rotations is isomorphic to $S_4$.

Similar argument for the existence of an epimorphism of $S_4$ onto $S_3$.
\end{proof}

\begin{exercise}
Show that the group of rotations of the dodecahedron has 60 elements. Using the fact that five cubes can be inscribed in a dodecahedron, or otherwise, show that it is isomorphic to $A_5$.
\end{exercise}
\begin{proof}
The dodecahedron has $12$ pentagonal faces so the group of rotations has $12\cdot 5=60$ elements.

The group of rotations is isomorphic to a subgroup of $S_5$ and the only subgroup of $S_5$ with $60$ elements is $A_5$.
\end{proof}

\begin{exercise}
A group series $(G_i)_{i=0}^r$ is an upper central series if $G_{i-1}/G_i=Z(G/G_i)$ for $1\leq i\leq r$. Show that $G$ possesses an upper central series if and only if $G$ is nilpotent.
\end{exercise}
\begin{proof}
$(\implies)$ Trivial.

$(\impliedby)$ Let $(G_i)_{i=0}^r$ be a central series. Define a series $(H_i)_{i=0}^r$ by $H_0=G$ and $H_i=[G_{i-1},G_{i-1}]$ for $1\leq i\leq r$.

We have $[g,h]\in G_i\forall g\in G,h\in G_{i-1}$
\end{proof}

\begin{exercise}
Let $z\in\mathbb{Q}$. Show that $z$ is not conjugate to $z'$ for any complex number $z'\neq z$.
\end{exercise}
\begin{proof}
Define $p\in\mathbb{Q}[x]$ by $p(x)=x-z$. Then $p(z)=0$ and $p(z')\neq 0 $ so $z$ and $z'$ are not conjugate.
\end{proof}

\begin{exercise}
Show that Gal$(f)$ is a subgroup of $S_k$.
\end{exercise}
\begin{proof}
$(\alpha_1,...,\alpha_k)$ is conjugate to itself so Id $\in S_k$.

Let $\sigma,\tau\in$ Gal$(f)$. Let $p\in\mathbb{Q}[x_1,...x_k]$ such that $p(\alpha_1,...,\alpha_k)=0$ so that $p(\alpha_{\sigma(1)},...,\alpha_{\sigma(k)})=0$ and $p(\alpha_{\tau(1)},...,\alpha_{\tau(k)})=0$. Define $q\in\mathbb{Q}[x_1,...x_k]$ by $q(x_1,...,x_k)=p(\tau(x_1),...,\tau(x_k))$. Then $q(\alpha_1,...,\alpha_k)=p(\alpha_{\tau(1)},...,\alpha_{\tau(k)})=0\implies q(\alpha_{\sigma(1)},...,\alpha_{\sigma(k)})=p(\alpha_{\tau\circ\sigma(1)},...,\alpha_{\tau\circ\sigma(k)})=0$.

Now let $p\in\mathbb{Q}[x_1,...x_k]$ such that $p(\alpha_{\tau\circ\sigma(1)},...,\alpha_{\tau\circ\sigma(k)})=0$. Define $q\in\mathbb{Q}[x_1,...x_k]$ by $q(x_1,...,x_k)=p(\tau(x_1),...,\tau(x_k))$. Then $q(\alpha_{\sigma(1)},...,\alpha_{\sigma(k)})=p(\alpha_{\tau\circ\sigma(1)},...,\alpha_{\tau\circ\sigma(k)})=0\implies q(\alpha_1,...,\alpha_k)=p(\alpha_{\tau(1)},...,\alpha_{\tau(k)})=0\implies p(\alpha_1,...,\alpha_k)=0$. Thus $\tau\circ\sigma\in$ Gal$(f)$.

Let $\sigma\in$ Gal$(f)$. Let $p\in\mathbb{Q}[x_1,...x_k]$. Define $q\in\mathbb{Q}[x_1,...x_k]$ by $q(x_1,...,x_k)=p(\sigma^{-1}(x_1),...,\sigma^{-1}(x_k))$. Then $p(\alpha_{\sigma^{-1}(1)},...,\alpha_{\sigma^{-1}(k)})=q(\alpha_1,...,\alpha_k)=0\iff q(\alpha_{\sigma(1)},...,\alpha_{\sigma(k)})=p(\alpha_1,...,\alpha_k)=0$ so $\sigma^{-1}\in$ Gal$(f)$.

Thus Gal$(f)\leqslant S_k$.
\end{proof}

\begin{exercise}
Prove that the only subring of a ring $R$ that is also an ideal is $R$ itself.
\end{exercise}
\begin{proof}
Let $S$ be a subring of $R$ which is also an ideal. $S$ contains $1$ so $r\cdot 1=r\in S\forall r\in R$ so $S=R$.
\end{proof}

\begin{exercise}
Show that the only ring in which $0=1$ is the trivial ring.
\end{exercise}
\begin{proof}
Let $R$ be a non-trivial ring with $0=1$ and let $a\in R$ be non-zero. Then $a=1\cdot a=0\cdot a=0$; a contradiction.
\end{proof}

\begin{exercise}
Show that $\mathbb{Z}$ is a principal ideal domain.
\end{exercise}
\begin{proof}
Let $I$ be an ideal in $\mathbb{Z}$. If $I=\{0\}$ then $I=\langle 0\rangle $. Otherwise, let $n$ be the smallest positive integer in $I$. Let $a\in I$. Then $a=kn+r$ for some $k\in\mathbb{Z}$ and $0\leq r<n$. $kn\in I$ so $r=a-kn\in I$. $n$ is the smallest positive integer in $I$ so $r=0$ meaning that $n$ divides $a$. Thus $I=\langle n\rangle$.
\end{proof}

\begin{exercise}
let $\phi:K\to L$ be a homomorphism of fields. Show that char $K=$ char $L$.
\end{exercise}
\begin{proof}
Suppose that $K$ has characteristic $0$. Then $m\cdot 1_K\neq 0\forall m>0$. Thus $\phi(m\cdot 1_K)=m\cdot 1_L\neq0\forall m>0$ by injectivity so $L$ has characteristic $0$.

Now suppose that $K$ has characteristic $p>0$. Then $p\cdot 1_L=\phi(p\cdot 1_K)=\phi(0)=0$. Now suppose that there exists an $m$ such that $0<m<p$ and $m\cdot 1_L=0$. $\phi$ is injective so $m\cdot 1_L=\phi(m\cdot 1_K)=0\implies m\cdot 1_K=0$; a contradiction. Thus $L$ has characteristic $p$.
\end{proof}

\begin{exercise}
Let $p$ be a prime and consider the field $\mathbb{F}_p(t)$ of rational expressions over $\mathbb{F}$. Show that $t$ has no $p$th root in $\mathbb{F}_p(t)$.
\end{exercise}

\begin{proof}
Suppose that $t$ has a $p$th root $a\in \mathbb{F}_p(t)$. Write $a$ as $\frac{r}{s}$ for $r,s\in\mathbb{F}_p[t]$. Then $t=(\frac{r}{s})^p\implies ts^p=r^p\implies 1+p\text{deg}(s)=p\text{deg}(r)\implies 1=p(\text{deg}(r)-\text{deg}(s))$. A contradiction.
\end{proof}

\begin{exercise}
Let $f$ be a quadratic polynomial over $\mathbb{Q}$. Prove that $\text{Gal}(f)$ is $S_2$ if $f$ has two distinct irrational roots, and trivial otherwise.
\end{exercise}
\begin{proof}
Let $f$ have distinct irrational roots $\alpha_1,\alpha_2$. Let $d$ be the discriminant of $f$, which is irrational. Since $d^2$ is rational, we can express any $\omega\in\mathbb{Q}(d)$ uniquely as $a+bd$ where $a,b\in\mathbb{Q}$. Define $\overline{\omega}$ as $a-bd$. Given $\nu=f+gd$, we have $\overline{\nu}+\overline{\omega}=(a+f)-(b+g)d=\overline{\nu+\omega}$. Furthermore, $\overline{\nu}\cdot\overline{\omega}=(a-bd)(f-gd)=af-(ag+bf)d+bgd^2$ and $\overline{\nu\cdot\omega}=\overline{af+(bf+ag)d+bgd^2}=af-(bf+ag)d+bgd^2$ so $\overline{\nu}\cdot\overline{\omega}=\overline{\nu\cdot\omega}$. Now let $p\in\mathbb{Q}[x_1,x_2]$. By induction we have that $p(\overline{x_1},\overline{x_2})=\overline{p(x_1,x_2)}$ so $p(\alpha_1,\alpha_2)=0\iff p(\overline{\alpha_1},\overline{\alpha_2})=\overline{0}=0\iff p(\alpha_2,\alpha_1)=0$. Thus $(\alpha_1,\alpha_2)$ and $(\alpha_2,\alpha_1)$ are conjugate so $\text{Gal}(f)=S_2$.

The case where both roots are rational was covered in an exercise. It's not possible for only one root to be rational by the quadratic formula so all cases are covered.
\end{proof}

\begin{exercise}
Let $R$ be a ring and let $\phi\colon1\to R$ be a homomorphism, where $1$ denotes the trivial ring. Prove that $R$ is trivial too and that $\phi$ is an isomorphism.
\end{exercise}
\begin{proof}
$1_R=\phi(1_1)=\phi(0_1)=0_R$ so $R$ is trivial. $\phi$ is injective and surjective so a ring isomorphism.
\end{proof}

\begin{exercise}
Let $f(t)=a_0+a_1t+...+a_nt^n\in\mathbb{Z}[t]$. Let $c/d$ be a rational root of $f$, where $c$ and $d$ are coprime integers. Prove the rational roots theorem: $c|a_0$ and $d|a_n$.
\end{exercise}
\begin{proof}
$f(c/d)=a_0+a_1c/d+...+a_n(c/d)^n=0\implies a_0d^n+a_1cd^{n-1}+...+a_nc^n=0$. Thus $a_0\equiv0\text{ mod }c$ and $a_n\equiv0\text{ mod }d$ as required.
\end{proof}

\begin{exercise}
Can $C_6$ act faithfully on a $4$-element set?
\end{exercise}
\begin{proof}
No. Let $S$ be the $4$-element set and suppose that $\Phi\colon C_6\to\text{Sym}(S):g\mapsto(x\mapsto g\cdot x)$ is injective, ie has trivial kernel. $1$ has order $6$ so $\Phi(1)$ also has order $6$. However, $\text{Sym}(S)$ has no element of order $6$; a contradiction.
\end{proof}

\begin{exercise}
Let $G$ be a finite group acting transitively on a nonempty set $X$. Prove that $|X|$ divides $|G|$.
\end{exercise}
\begin{proof}
Let $x\in X$. Then $|G|=|\text{Orb}(x)|\cdot|\text{Stab}(x)|=|X|\cdot|\text{Stab}(x)|$ so $|X|$ divides $|G|$.
\end{proof}

\begin{exercise}
Let $R$ be a ring and let $I_0\subseteq I_1\subseteq...$ be ideals of $R$. Prove that $\bigcup_{n=0}^\infty I_n$ is an ideal of $R$.
\end{exercise}
\begin{proof}
Let $a,b\in \bigcup_{n=0}^\infty I_n$. We have that $a\in I_k$ for some $k$. Then given any $r\in R$, we have $ra\in I_k\subseteq \bigcup_{n=0}^\infty I_n$. We also have that $b\in I_l$ for some $l$. Let $m=\text{max}(k,l)$ so that $a,b\in I_m$. Then $b-a\in I_m\subseteq\bigcup_{n=0}^\infty I_n$. Thus $\bigcup_{n=0}^\infty I_n$ is an ideal.
\end{proof}

\begin{exercise}
Let $R$ be a principle ideal domain and let $I_0\subseteq I_1\subseteq ...$ be ideals of $R$. Prove that there is some $n\geq0$ such that $I_n=I_{n+1}=I_{n+2}=...$
\end{exercise}
\begin{proof}
There exists some $a\in R$ such that $\bigcup_{n=0}^\infty I_n=\langle a\rangle$. There exists some $k\in\mathbb{N}$ such that $a\in I_k$ implying that $\bigcup_{n=0}^\infty I_n= I_n\forall n\geq k$.
\end{proof}

\begin{exercise}
Let $R$ be an integral domain. Let $r,s\in R$ with $r\neq 0$ and $s$ not a unit. Prove that $\langle rs\rangle$ is a proper subset of $\langle r \rangle$.
\end{exercise}
\begin{proof}
Clearly $\langle rs\rangle\subseteq\langle r\rangle$. Now suppose that $\langle rs\rangle=\langle r\rangle$. We then have some $a\in R$ such that $ars=r\implies r(as-1)=0$. $r\neq 0$ so $as=1$. But $s$ is not a unit; a contradiction.
\end{proof}

\begin{exercise}
Let $R$ be a principle ideal domain. Let $r\in R$ be neither $0$ not a unit. Prove that some irreducible divides $r$.
\end{exercise}
\begin{proof}
Suppose that no irreducible divides $r$. Then writing $r_0=r$, we have $r_0=r_1s_1$ for some non-units $r_1$ and $s_1$. Similarly, no irreducible divides $r_1$ so we can express $r_1$ as $r_1=r_2s_2$ for some non-units $r_2$ and $s_2$. Continue this ad infinitum such that $r_i=r_{i+1}s_{i+1}$ for non-units $r_{i+1}s_{i+1}$ with $r_i$ not divisible by an irreducible. We then have $\langle r_0\rangle\subset\langle r_1\rangle\subset\langle r_2\rangle\subset...$. However, there exists some $i$ such that $\langle r_i\rangle=\langle r_{i+1}\rangle=\langle r_{i+2}\rangle=...$; a contradiction.
\end{proof}

\begin{exercise}
For $f\in K[t]$, there is a unique homomorphism $\theta_f\colon K[t]\to K[t]$ such that $\theta_f(t)=f$ and $\theta_f(a)=a$ for all $a\in K$. Which result guarantees this?
\end{exercise}
\begin{proof}
Universal property of the polynomial ring. The homomorphism is given by $\theta_f(\sum_i a_it^i)=\sum_i a_if^i$
\end{proof}
\begin{exercise}
For $f,g\in K[t]$, what is $\theta_f(g)$ in explicit terms?
\end{exercise}
\begin{proof}
$g(f)$ and it has degree $\text{deg}(f)\text{deg}(g)$.
\end{proof}

\begin{exercise}
For $f_1,f_2\in\mathbb{K}[t]$, what can you say about the composite homomorphism\[\mathbb{K}[t]\overset{\theta_{f_1}}{\to}\mathbb{K}[t]\overset{\theta_{f_2}}{\to}\mathbb{K}[t]?\]
\end{exercise}
\begin{proof}
$\text{deg}(\theta_{f_2}\circ\theta_{f_1}(g))=\text{deg}(f_1)\text{deg}(f_2)\text{deg}(g)$.
\end{proof}
\begin{exercise}
Using the previous parts, find all the isomorphisms $\mathbb{K}[t]\to\mathbb{K}[t]$ over $\mathbb{K}$.
\end{exercise}
\begin{proof}
Any isomorphism over $\mathbb{K}$ is of the form $\theta_f$ for some $f\in\mathbb{K}[t]$, with $f$ necessarily having degree $1$ and hence linear. Let $f$ be given as $at+b$ for some $a,b\in\mathbb{K}$ with $a\neq0$. Then $\theta_f(g)=g(at+b)\forall g\in\mathbb{K}[t]$. $\theta_{\frac{t-b}{a}}\circ\theta_{f}(g)=\theta_{\frac{t-b}{a}}(g(at+b))=g(a\frac{t-b}{a}+b)=g\forall g\in \mathbb{K}[t]$ and $\theta_f\circ\theta_{\frac{t-b}{a}}(g)=\theta_f(g(\frac{t-b}{a}))=g(\frac{at+b-b}{a})=g\forall g\in\mathbb{K}[t]$ and so $\theta_f$ is an isomorphism. Hence all isomorphisms $\mathbb{K}[t]\to\mathbb{K}[t]$ over $\mathbb{K}$ are of the form $\theta_{at+b}$ for non-zero $a$.
\end{proof}

\begin{exercise}
Let $f(t)=t^4+t^3+t^2+t+1$, which has roots $\omega,\omega^2,\omega^3,\omega^4$ where $\omega=e^{2\pi i/5}$. One of the elements of $\text{Gal}(f)$ is $\sigma$ given by $\omega\mapsto\omega^2\mapsto\omega^4\mapsto\omega^3\mapsto\omega$. Prove that $\text{Gal}(f)$ is generated by $\sigma$, and deduce that $\text{Gal}(f)\cong C_4$.
\end{exercise}
\begin{proof}

\end{proof}

\begin{exercise}
prove that $\text{cos}(\pi/9)$ is algebraic over $\mathbb{Q}$, and find its minimal polynomial.
\end{exercise}
\begin{proof}
We have $\text{cos}(3\theta)=4\text{cos}^3(\theta)-3\text{cos}(\theta)$ and so $1/2=\text{cos}(\pi/3)=4\text{cos}^3(\pi/9)-3\text{cos}(\pi/9)$. Thus $\text{cos}(\pi/9)$ is algebraic over $\mathbb{Q}$ and a solution of $f=8t^3-6t-1$. $5$ is a prime which does not divide $8$ and $\overline{f}=3t^3-t-1\in\mathbb{F}_5[t]$ is irreducible so $f$ is irreducible over $\mathbb{Q}$ by the Mod $p$ method and thus the minimal polynomial.
\end{proof}

\begin{exercise}
Show that for every $n\geq1$, there exists an extension of $\mathbb{Q}$ of degree $n$.
\end{exercise}
\begin{proof}
Consider the polynomial $t^n-2$. The prime $2$ does not divide the leading coefficient and does divide all other coefficients; furthermore $2^2$ does not divide the constant term so $t^n-2$ is irreducible over $\mathbb{Q}$ by Eisenstein's criterion. $\sqrt[n]{2}$ then has minimal polynomial $t^n-2$ so $\mathbb{Q}(\sqrt[n]{2}):\mathbb{Q}$ has degree $n$.
\end{proof}

\begin{exercise}
Which of the following polynomials are irreducible over $\mathbb{Q}$?
\begin{enumerate}[label=(\roman*)]
    \item $1+2t-5t^3+2t^6$
    \item $4-3t-2t^2$
    \item $4-13t-2t^3$
    \item $1+t+t^2+t^3+t^4+t^5$
    \item $1+t+t^2+t^3+t^4+t^5+t^6$
    \item $2.2+3.3t-1.1t^3+t^7$
    \item $1+t^4$
\end{enumerate}
\end{exercise}
\begin{proof}
\begin{enumerate}[label=(\roman*)]
    \item Reducible since $1$ is a root
    \item The discriminant is $9+4*4*2=41$ which is not a square so the polynomial is irreducible.
    \item Taken modulo $3$, the polynomial becomes $1-t-2t^3$ which has no solutions modulo $3$ so the polynomial is irreducible by the Mod $p$ method.
    \item The polynomial is $\frac{t^6-1}{t-1}=\frac{(t^3-1)(t^3+1)}{t-1}=\frac{(t-1)(t^2+t+1)(t^3+1)}{t-1}=(t^2+t+1)(t^3+1)$ so not irreducible
    \item Yes, since it's $\Phi_7$ and cyclotomic polynomials are irreducible.
    \item $10$ is a unit so the polynomial is irreducible if and only if $f=22+33t-11t^3+10t^7$ is irreducible. The prime $11$ divides each coefficient other than the leading one and $100$ does not divide the constant term so $f$ is irreducible by Eisenstein's criterion.
    \item The polynomial is irreducible modulo $3$ so is irreducible by the Mod $p$ method.
\end{enumerate}
\end{proof}

\begin{exercise}
Let $\mathbb{M}:\mathbb{K}$ be a field extension. Show that $\mathbb{K}(Y\cup Z)=(\mathbb{K}(Y))(Z)$ whenever $Y,Z\subseteq \mathbb{M}$
\end{exercise}
\begin{proof}
$\mathbb{K}(Y\cup Z)=\bigcap \text{Subfields of }\mathbb{M}\text{ containing }\mathbb{K}\cup(Y\cup Z)=\bigcap \text{Subfields of }\mathbb{M}\text{ containing }(\mathbb{K}\cup Y)\cup Z=(\mathbb{K}(Y))(Z)$
\end{proof}

\begin{exercise}
Let $\mathbb{M}:\mathbb{K}$ be a field extension and $\alpha,\beta\in \mathbb{M}$. Call $\alpha$ and $\beta$ conjugate over $\mathbb{K}$ if for all $p\in \mathbb{K}[t]$, we have $p(\alpha)=0\iff p(\beta)=0$.
\begin{enumerate}[label=(\roman*)]
    \item Prove that $\alpha$ and $\beta$ are conjugate over $\mathbb{K}$ if and only if either both are transcendental or both are algebraic and they have the same minimal polynomial.
    \item Show that if there exists an irreducible polynomial $p\in \mathbb{K}[t]$ such that $p(\alpha)=0=p(\beta)$, then $\alpha$ and $\beta$ are conjugate over $\mathbb{K}$.
    \item Show that if $\alpha$ and $\beta$ are conjugate over $\mathbb{K}$ then $\mathbb{K}(\alpha)\cong\mathbb{K}(\beta)$ over $\mathbb{K}$.
    \item Let $p$ be a prime number and put $\omega=e^{2\pi i/p}$. Prove that $\omega,...,\omega^{p-1}$ are conjugate over $\mathbb{Q}$.
    \item Prove that $\mathbb{Q}(\pi)\cong\mathbb{Q}(e)$ over $\mathbb{Q}$.
\end{enumerate}
\end{exercise}

\begin{proof}
    \begin{enumerate}[label=(\roman*)]
        \item $(\impliedby)$If both are transcendental then they're vacuously conjugate. Suppose both are algebraic with the same minimal polynomial $m$. $\text{ker}(\text{ev}_\alpha)=\langle m\rangle=\text{ker}(\text{ev}_\beta)$ so $p(\alpha)=0\iff p\in\text{ker}(\text{ev}_\alpha)\iff p\in \text{ker}(\text{ev}_\beta)\iff p(\beta)=0$.
        
        $(\implies)$ Suppose $\alpha$ and $\beta$ are conjugate over $\mathbb{K}$. Then $\text{ker}(\text{ev}_\alpha)=\text{ker}(\text{ev}_\beta)$. If the ideals are not trivial then $\alpha$ and $\beta$ are algebraic with the same minimal polynomials (the unique monic polynomial which generates the ideal). If the ideals are trivial, then $\alpha$ and $\beta$ are transcendental.
        \item They would then be algebraic with the same minimal polynomial so conjugate.
        \item First consider the case where $\alpha$ and $\beta$ are algebraic with minimal polynomial $m$. Then $\mathbb{K}(\alpha)\cong \mathbb{K}[t]/\langle m\rangle\cong \mathbb{K}(\beta)$, with each isomorphism fixing $\mathbb{K}$ pointwise.
        Now consider the case when they're both transendental. Then $\mathbb{K}(\alpha)\cong\mathbb{K}(x)\cong \mathbb{K}(\beta)$, with the isomorphisms again fixing $\mathbb{K}$ pointwise.
        \item They're all roots of the irreducible (over $\mathbb{Q}$) polynomial $\Phi_p$ so are conjugate over $\mathbb{Q}$.
        \item Both $\pi$ and $e$ are transcendental over $\mathbb{Q}$ so apply part $(iii)$.
    \end{enumerate}
\end{proof}

\begin{exercise}
Let $\mathbb{M}:\mathbb{L}:\mathbb{K}$ be field extensions, which you may not assume to be finite.
\begin{enumerate}[label=(\roman*)]
    \item Let $\alpha\in \mathbb{M}$. Prove that if $\alpha$ is algebraic over $\mathbb{L}$ and $\mathbb{L}$ is algebraic over $\mathbb{K}$ then $\alpha$ is algebraic over $\mathbb{K}$.
    \item Deduce that if $\mathbb{M}:\mathbb{L}$ and $\mathbb{L}:\mathbb{K}$ are algebraic then so is $\mathbb{M}:\mathbb{K}$.
\end{enumerate}
\end{exercise}

\begin{proof}
\begin{enumerate}[label=(\roman*)]
    \item Let $p=a_0+a_1x+...+a_nx^n\in\mathbb{L}[x]$ be a non-zero polynomial such that $p(\alpha)=0$. Note that $p\in \mathbb{K}(a_0,...,a_n)[x]$. $\mathbb{K}(a_0,...,a_n):\mathbb{K}$ is finite and $\mathbb{K}(a_0,...,a_n,\alpha):\mathbb{K}(a_0,...,a_n)$ is simple algebraic, so finite, hence $\mathbb{K}(a_0,...,a_n,\alpha):\mathbb{K}$ is finite so algebraic. Thus $\alpha$ is algebraic over $\mathbb{K}$.
    \item Every $\alpha\in \mathbb{M}$ is algebraic over $\mathbb{K}$ so $\mathbb{M}:\mathbb{K}$ is algebraic.
\end{enumerate}
\end{proof}

\begin{exercise}
Prove that $\overline{\mathbb{Q}}$, the subfield of $\mathbb{C}$ consisting of the complex numbers algebraic over $\mathbb{Q}$, is algebraically closed.
\end{exercise}
\begin{proof}
Let $f\in \overline{\mathbb{Q}}[x]$ be a non-zero polynomial and let $\mathbb{F}$ be a splitting field of $f$. $\mathbb{F}:\overline{\mathbb{Q}}$ is algebraic and $\overline{\mathbb{Q}}:\mathbb{Q}$ is algebraic so $\mathbb{F}:\mathbb{Q}$ is algebraic, so every root of $f$ is in $\overline{\mathbb{Q}}$.
\end{proof}

\begin{exercise}
\begin{enumerate}[label=(\roman*)]
    \item Here I will write $\langle X\rangle$ for the subfield generated by a subset $X$ of a field $\mathbb{K}$. Show that for all $X\subseteq\mathbb{K}$ and homomorphisms of fields $\phi:\mathbb{K}\to\mathbb{L}$,\[\phi\langle X\rangle=\langle\phi X\rangle.\]
    \item Let $\mathbb{M}:\mathbb{K}$ and $\mathbb{M}':\mathbb{K}$ be field extensions, and let $\phi:\mathbb{M}\to\mathbb{M}'$ be a homomorphism over $\mathbb{K}$. Show that $\phi(\mathbb{K}(Y))=\mathbb{K}(\phi Y)$ for all subsets $Y$ of $\mathbb{M}$
\end{enumerate}
\end{exercise}

\begin{proof}
\begin{enumerate}[label=(\roman*)]
    \item $\phi X\subseteq \phi\langle X\rangle\implies \langle\phi X\rangle\subseteq\langle\phi\langle X\rangle\rangle=\phi\langle X\rangle$ since $\phi\langle X\rangle$ is already a field. Suppose that $Y$ is a subfield of $\mathbb{L}$ containing $\phi X$. Then $X=\phi^{-1}\phi X\subseteq\phi^{-1}Y$. $\phi^{-1}Y$ is a field so $\langle X\rangle\subseteq\phi^{-1}Y\implies \phi\langle X\rangle\subseteq Y$. Thus $\langle\phi X\rangle=\phi\langle X\rangle$.
    \item $\phi(\mathbb{K}(Y))=\phi(\langle\mathbb{K}\cup Y\rangle)=\langle\phi(\mathbb{K}\cup Y)\rangle=\langle\phi(\mathbb{K})\cup\phi(Y)\rangle=\langle\mathbb{K}\cup\phi(Y)\rangle=\mathbb{K}(\phi Y)$.
\end{enumerate}
\end{proof}

\begin{exercise}
Let $f$ be a nonconstant polynomial over $\mathbb{Z}$. Prove that $f$ is primitive and irreducible over $\mathbb{Q}\iff f$ is irreducible over $\mathbb{Z}$.
\end{exercise}
\begin{proof}
$(\impliedby)$ Gauss' lemma.

$(\implies)$ Suppose that $f$ is primitive and irreducible over $\mathbb{Q}$ but is reducible over $\mathbb{Z}$. Then $f=gh$ for some $g,h\in\mathbb{Z}[x]$, neither of which are $0$ or $1$ (the only unit). If both have degree at least $1$ then $f$ is reducible over $\mathbb{Q}$. Thus one is a constant other than $0$ or $1$. But then $f$ is an integer multiple other than $0$ or $1$ of a polynomial over $\mathbb{Z}$ and so isn't primitive; a contradiction.
\end{proof}

\begin{exercise}
\begin{enumerate}[label=(\roman*)]
    \item Let $\mathbb{K}$ be a field and $a\in\mathbb{K}$. Show that\[[\mathbb{K}(\sqrt{a}):\mathbb{K}] =\begin{cases}
       1 &\quad\text{if }a\text{ has a root in }\mathbb{K} \\
       2 &\quad\text{otherwise} \\ 
     \end{cases}\]
     \item Let $\mathbb{L}$ be a field with char $\mathbb{L}\neq2$, and let $a,b,c,\alpha\in\mathbb{L}$ with $a\neq0$. Suppose that $a\alpha^2+b\alpha+c=0$, Prove that $b^2-4ac$ has a square root $\sigma$ in $\mathbb{L}$, and that\[\alpha\in\{\frac{-b+\sigma}{2a},\frac{-b-\sigma}{2a}\}.\]
     \item Let $\mathbb{L}:\mathbb{K}$ be a field extension of degree $2$, with char $\mathbb{K}\neq 2$. Prove that $\mathbb{L}\cong\mathbb{K}(\sqrt{d})$ for some $d\in\mathbb{K}$.
\end{enumerate}
\end{exercise}

\begin{proof}
\begin{enumerate}[label=(\roman*)]
    \item Let $m\in \mathbb{K}[x]$ be the minimal polynomial of $\sqrt{a}$. If $\sqrt{a}\in\mathbb{K}$ obvious. Otherwise $m=x^2-a$ so has degree $2$.
    \item $\alpha^2+\frac{b}{a}\alpha+\frac{c}{a}=0\implies (\alpha+\frac{b}{2a})^2-\frac{b^2}{4a^2}+\frac{c}{a}=0\implies (\alpha+\frac{b}{2a})^2=\frac{b^2-4ac}{4a^2}\implies b^2-4ac=((2a)(\alpha+\frac{b}{2a}))^2$ so $b^2-4ac$ has a square root of $\sigma=(2a)(\alpha+\frac{b}{2a})$ in $\mathbb{L}$. The rest is just computation.
    \item We have that $\mathbb{L}=\mathbb{K}\oplus\alpha\mathbb{K}$ with $\alpha$ having a minimal polynomial $m\in\mathbb{K}$ of degree $2$. Let $m=ax^2+bx+c$. Then $\alpha=\frac{-b+\sqrt{b^2-4ac}}{2a}$ where $\sqrt{b^2-4ac}$ is the correctly chosen square root. Thus $\mathbb{L}=\mathbb{K}(\alpha)=\mathbb{K}(\sqrt{b^2-4ac})$.
\end{enumerate}
\end{proof}

\begin{exercise}
Prove that the field extension $\overline{\mathbb{Q}}:\mathbb{Q}$ is not finite. Deduce that $\overline{\mathbb{Q}}:\mathbb{Q}$ is not even finitely generated.
\end{exercise}
\begin{proof}
Suppose $\overline{\mathbb{Q}}:\mathbb{Q}$ is finite with degree $n$. There is an irreducible polynomial $f\in\mathbb{Q}[x]$ of degree $n+1$. Let $\alpha\in\overline{\mathbb{Q}}$ be a root of $f$. Then $n=[\overline{\mathbb{Q}}:\mathbb{Q}]=[\overline{\mathbb{Q}}:\mathbb{Q}(\alpha)][\mathbb{Q}(\alpha):\mathbb{Q}]\geq [\mathbb{Q}(\alpha):\mathbb{Q}]=n+1$; a contradiction. Hence $\overline{\mathbb{Q}}:\mathbb{Q}$ cannot be both finitely generated and algebraic. $\overline{\mathbb{Q}}:\mathbb{Q}$ is clearly algebraic so it isn't finitely generated.
\end{proof}

\begin{exercise}
A field extension $\mathbb{M}:\mathbb{K}$ is simple algebraic if there exists $\alpha\in \mathbb{M}$ such that $\mathbb{M}=\mathbb{K}(\alpha)$ and $\alpha$ is algebraic over $\mathbb{K}$. Prove that $\mathbb{M}:\mathbb{K}$ is simple algebraic if and only if it is simple and algebraic.
\end{exercise}
\begin{proof}
$(\implies)$ By definition.

$(\impliedby)$ $\mathbb{M}=\mathbb{K}(\alpha)$ for some $\alpha\in\mathbb{M}$. But $\mathbb{M}:\mathbb{K}$ is algebraic so $\alpha$ is as well.
\end{proof}

\begin{exercise}
Let $\mathbb{Q}(t_1,t_2,...)$ be the field of rational expressions in countably infinitely many symbols $t_1,t_2,...$ (an element is a ratio of polynomials in these symbols, and can involve only finitely many of them). it has a subfield $\mathbb{Q}(t_2,t_3,...)$. So we have extensions\[\mathbb{Q}(t_1,t_2,...):\mathbb{Q}(t_2,t_3,...),\mathbb{Q}(t_2,t_3,...):\mathbb{Q}(t_2,t_3,...).\]Prove that the fields $\mathbb{Q}(t_1,t_2,...)$ and $\mathbb{Q}(t_2,t_3,...)$ are isomorphic, but not isomorphic over $\mathbb{Q}(t_2,t_3,...)$.
\end{exercise}
\begin{proof}
Consider $\phi:\mathbb{Q}(t_1,t_2,...)\to\mathbb{Q}(t_2,t_3,...)$ given by $\phi(t_i)=t_{i+1}$ which is clearly an injective homomorphism between fields with inverse given by $\phi^{-1}(t_i)=t_{i-1}$ so is an isomorphism.

$[\mathbb{Q}(t_1,t_2,...):\mathbb{Q}(t_2,t_3,...)]=\infty$ whereas $[\mathbb{Q}(t_2,t_3,...):\mathbb{Q}(t_2,t_3,...)]=1$ so they can't be isomorphic (since isomorphic vector spaces have the same dimension).
\end{proof}

\begin{exercise}
Let $\mathbb{M}:\mathbb{K}$ be a finite extension. Prove that every homomorphism $\mathbb{M}\to\mathbb{M}$ over $\mathbb{K}$ is an automorphism of $\mathbb{M}$ over $\mathbb{K}$.
\end{exercise}
\begin{proof}
Let $\phi:\mathbb{M}\to\mathbb{M}$ be a homomorphism over $\mathbb{K}$. By the rank-nullity theorem, $[\mathbb{M}:\mathbb{K}]=\text{rank}(\phi)+\text{nullity}(\phi)=\text{rank}(\phi)$ since homomorphisms between fields are injective. Thus $\text{Im}(\phi)\leqslant \mathbb{M}$ and $\text{dim}(\text{Im}(\phi))=\text{dim}(\mathbb{M})$ so $\phi$ is surjective so an isomorphism.
\end{proof}

\begin{exercise}
Let $\mathbb{K},\mathbb{L}$ and $\mathbb{M}$ be subfields of a field $\mathbb{N}$, with $\mathbb{K}\subseteq\mathbb{L}$ and $\mathbb{K}\subseteq\mathbb{M}$. Here you will prove the 'diamond inequality', $[\mathbb{L}\mathbb{M}:\mathbb{L}]\leq[\mathbb{M}:\mathbb{K}]$.
\begin{enumerate}[label=(\roman*)]
    \item First suppose that $\mathbb{M}=\mathbb{K}(\beta)$ for some $\beta$ algebraic over $\mathbb{K}$. Show that $\mathbb{L}\mathbb{M}=\mathbb{L}(\beta)$, then deduce that $[\mathbb{L}\mathbb{M}:\mathbb{L}]\leq[\mathbb{M}:\mathbb{K}]$.
    \item Now prove the diamond inequality when $\mathbb{M}=\mathbb{K}(\beta_1,...,\beta_n)$ for some $\beta_1,...,\beta_n$ algebraic over $\mathbb{K}$,
    \item Finally, prove the diamond inequality in full generality.
\end{enumerate}
\end{exercise}

\begin{proof}
\begin{enumerate}[label=(\roman*)]
    \item $\mathbb{L}\subseteq\mathbb{L}(\beta)$ and $\mathbb{M}=\mathbb{K}(\beta)\subseteq\mathbb{L}(\beta)$ since $\mathbb{K}\subseteq\mathbb{L}$ so $\mathbb{L}\cup\mathbb{M}\subseteq\mathbb{L}(\beta)$. Now let $\mathbb{I}$ be a subfield of $\mathbb{N}$ containing $\mathbb{L}\cup\mathbb{M}$. $\mathbb{I}$ then contains $\mathbb{L}$ and $\beta$ so $\mathbb{L}(\beta)\subseteq\mathbb{I}$. Thus $\mathbb{L}\mathbb{M}=\mathbb{L}(\beta)$. Then $[\mathbb{L}\mathbb{M}:\mathbb{L}]=[\mathbb{L}(\beta):\mathbb{L}]\leq\text{deg}(m)=[\mathbb{M}:\mathbb{K}]$ where $m$ is the minimal polynomial of $\beta$ over $\mathbb{K}$.
    \item We proceed by induction on $n$. The base case was covered above. Assume true for $n=k$. For $n=k+1$, we have $\mathbb{M}=\mathbb{K}(\beta_1,...\beta_k)(\beta_{k+1})$, $\mathbb{L}\mathbb{K}(\beta_1,...,\beta_k)=\mathbb{L}(\beta_1,...,\beta_k)$ and $[\mathbb{L}\mathbb{K}(\beta_1,...,\beta_k):\mathbb{L}]\leq[\mathbb{K}(\beta_1,...,\beta_k):\mathbb{K}]$. $\mathbb{L}\mathbb{M}=\mathbb{L}(\mathbb{K}(\beta_1,...,\beta_k)\cup\{\beta_{k+1}\})=(\mathbb{L}\mathbb{K}(\beta_1,...,\beta_k))(\beta_{k+1})=\mathbb{L}(\beta_1,...,\beta_k)(\beta_{k+1})=\mathbb{L}(\beta_1,...,\beta_{k+1})$. Thus $[\mathbb{L}\mathbb{M}:\mathbb{L}]=[\mathbb{L}(\beta_1,...,\beta_{k+1}):\mathbb{L}]=[\mathbb{L}(\beta_1,...,\beta_{k+1}):\mathbb{L}(\beta_1,...,\beta_k)][\mathbb{L}(\beta_1,...,\beta_k):\mathbb{L}]\leq[\mathbb{L}(\beta_1,...,\beta_{k+1}):\mathbb{L}(\beta_1,...,\beta_k)][\mathbb{K}(\beta_1,...,\beta_k):\mathbb{K}]\leq [\mathbb{K}(\beta_1,...,\beta_{k+1}):\mathbb{K}(\beta_1,...,\beta_k)][\mathbb{K}(\beta_1,...,\beta_k):\mathbb{K}]=[\mathbb{K}(\beta_1,...,\beta_{k+1}):\mathbb{K}]=[\mathbb{M}:\mathbb{K}]$ as required.
    \item If $[\mathbb{M}:\mathbb{K}]=\infty$ then clearly the inequality holds. Otherwise, there are finitely many $\beta_1,...,\beta_n$ algebraic over $\mathbb{K}$ such that $\mathbb{M}=\mathbb{K}(\beta_1,...,\beta_n)$ and so part $(ii)$ applies.
\end{enumerate}
\end{proof}

\begin{exercise}
Let $F$ be the set of real numbers $z$ such that $z$ is a coordinate of some point in $\mathbb{R}^2$ constructible by ruler and compass from $(0,0)$ and $(1,0)$. ('A coordinate' means either the $x$-coordinate or the $y$-coordinate.) Prove that $F$ is a subfield of $\mathbb{R}$, and, moreover, that $F$ is the smallest subfield of $\mathbb{R}$ with the property that $z^2\in F\implies z\in F$ for all $z\in\mathbb{R}$.
\end{exercise}

\begin{exercise}
Let $\mathbb{M}:\mathbb{L}:\mathbb{K}$ be field extensions, with $\mathbb{M}:\mathbb{K}$ finite and normal. Prove that there is a smallest subfield $\mathbb{L}'$ of $\mathbb{M}$ such that $\mathbb{L}\subseteq\mathbb{L}'$ and $\mathbb{L}':\mathbb{K}$ is normal. Here 'smallest' means that $\mathbb{L}'\subseteq\mathbb{L}''$ for any other subfield $\mathbb{L}''$ with the same properties.
\end{exercise}
\begin{proof}
We have that $\mathbb{L}=\mathbb{K}(a_1,...,a_n)$ for $a_1,...,a_n$ algebraic over $\mathbb{K}$. Let $m_1,...,m_n\in\mathbb{K}[x]$ be their minimal polynomials. Then define $\mathbb{L}'$ to be the splitting field of $m_1...m_n$. Clearly $\mathbb{L}\subseteq\mathbb{L}'$, and $\mathbb{L}'\subseteq\mathbb{M}$ since $\mathbb{M}$ is normal and so contains all the roots of all irreducible polynomials over $\mathbb{K}$ with a root in $\mathbb{M}$. $\mathbb{L}'$ is also normal since all splitting fields are normal. Now suppose $\mathbb{L}''$ is another subfield with the same properties. Then $\mathbb{L}''$ will contain $a_1,...,a_n$ and so contain all the roots of $m_1,...,m_n$ and so contain $\mathbb{L}'$.
\end{proof}


I was wondering if an irreducible polynomial having a non-solvable Galois group implies that every root is non-radical. The answer is yes. To prove this, it suffices to prove that if an irreducible polynomial has a radical root, then every root is radical.

\begin{lemma}
Let $\mathbb{L}/\mathbb{K}$ be an algebraic extension, where $\mathbb{L}=\mathbb{K}(\alpha)$ for $\alpha\in\mathbb{L}$ with minimal polynomial $m\in\mathbb{K}[x]$. Let $\mathbb{F}$ be a splitting field of $m$. Then $\mathbb{F}/\mathbb{K}$ is the normal closure of $\mathbb{L}/\mathbb{K}$.
\end{lemma}
\begin{proof}
$\mathbb{F}/\mathbb{K}$ is normal since $\mathbb{F}$ is the splitting field of a polynomial over $\mathbb{K}$. We now show that $\mathbb{F}/\mathbb{K}$ is minimal. Let $\mathbb{M}$ be a subfield of $\mathbb{F}$ containing $\mathbb{L}$ with $\mathbb{M}/\mathbb{K}$ normal. $m$ must then split into linear factors over $\mathbb{M}$, since $\alpha\in\mathbb{M}$, and so $\mathbb{M}$ contains every root of $m$ and hence contains $\mathbb{F}$. Thus $\mathbb{M}=\mathbb{F}$.
\end{proof}

\begin{lemma}
The normal closure of a finite radical extension is a radical extension.
\end{lemma}

\begin{theorem}
If a root of an irreducible polynomial is radical, then every root is radical.
\end{theorem}
\begin{proof}
Let $\mathbb{L}/\mathbb{K}$ be an extension contained in a finite radical extension $\mathbb{M}/\mathbb{K}$. Then the normal closure of $\mathbb{L}/\mathbb{K}$ is contained in the normal closure of $\mathbb{M}/\mathbb{K}$ which is radical by the above lemma. Now suppose that $\mathbb{L}=\mathbb{K}(\alpha)$ for $\alpha\in\mathbb{L}$ algebraic over $\mathbb{K}$ with minimal polynomial $m\in\mathbb{K}[x]$ and $\mathbb{M}/\mathbb{K}$ is a finite radical extension containing $\mathbb{L}/\mathbb{K}$; ie, $m$ has a radical root. Then the spitting field of $m$, which is the normal closure of $\mathbb{L}/\mathbb{K}$, is contained in a radical extension (the normal closure of $\mathbb{M}/\mathbb{K}$) and hence every root of $m$ is radical.
\end{proof}


\begin{exercise}
Let $\mathbb{M}:\mathbb{K}$ be a field extension. Let $0\neq f\in\mathbb{K}[t]$, and let $\alpha\in\mathbb{M}$ be a root of $f$; then $f(t)=(t-\alpha)g(t)$ for some $g(t)\in\mathbb{K}(\alpha)[t]$. Prove that $\mathbb{M}$ is a splitting field of $g$ over $\mathbb{K}(\alpha)\iff\mathbb{M}$ is a splitting field of $f$ over $\mathbb{K}$.
\end{exercise}
\begin{proof}
$(\implies)$ We have that $\mathbb{M}=\mathbb{K}(\alpha)(a_1,...,a_n)$ where $a_1,...,a_n$ are the roots of $g$. But then $\mathbb{M}=\mathbb{K}(\alpha,a_1,...,a_n)$ where $alpha,a_1,...,a_n$ are the roots of $f$ and so $\mathbb{M}$ is a splitting field of $f$ over $\mathbb{K}$.

$(\impliedby)$ We have that $\mathbb{M}=\mathbb{K}(\alpha,a_1,...,a_n)$ where $a_1,...,a_n$ are the other roots of $f$ and so $\mathbb{M}=\mathbb{K}(\alpha)(a_1,...,a_n)$ where $a_1,...,a_n$ are the roots of $g$.
\end{proof}

\begin{exercise}
Let $\mathbb{K}$ be a field and let $f\in\mathbb{K}[t]$ be an irreducible polynomial.
\begin{enumerate}[label=(\roman*)]
    \item Prove that the order of $\text{Gal}_\mathbb{K}(f)$ is divisible by the number of distinct roots of $f$ in its splitting field.
    \item Deduce that if $\text{Char}(\mathbb{K})=0$ then deg$(f)$ divides $|\text{Gal}_\mathbb{K}(f)|$.
\end{enumerate}
\end{exercise}
\begin{proof}
\begin{enumerate}[label=(\roman*)]
    \item let $a_1,...,a_n$ be the distinct roots of $f$. $\text{Gal}_\mathbb{K}(f)$ acts transitively on the roots so $n$ divides the order of $\text{Gal}_\mathbb{K}(f)$ by the orbit-stabilizer theorem.
    \item $f$ is separable so $\text{deg}(f)=n$.
\end{enumerate}
\end{proof}


\begin{exercise}
Let $\mathbb{M}:\mathbb{K}$ be a finite normal separable field extension. let $H$ be a subgroup of $G=\text{Gal}(\mathbb{M}:\mathbb{K})$. Prove that $H$ is a normal subgroup of $G$ if and only if Fix$(H)$ is a normal extension of $\mathbb{K}$, and that if these conditions hold them $G/H\cong\text{Gal}(\text{Fix}(H):\mathbb{K})$.
\end{exercise}
\begin{proof}
Since $\mathbb{M}:\mathbb{K}$ is a Galois extension, we have that $H=\text{Aut}_\mathbb{L}\mathbb{M}$ for some unique field $\mathbb{L}$ such that $\mathbb{K}\subseteq\mathbb{L}\subseteq\mathbb{M}$. Then by the fundamental theorem of Galois theory, $H=\text{Aut}_\mathbb{L}\mathbb{M}$ is a normal subgroup of $G$ if and only if $\mathbb{L}=\text{Fix}(H)$ is a normal extension, and furthermore $\text{Gal}(\text{Fix}(H):\mathbb{K})=\text{Gal}(\mathbb{L}:\mathbb{K})\cong G/\text{Aut}_\mathbb{L}\mathbb{M}=G/H$.
\end{proof}

\begin{exercise}
Prove that every field extension of degree $2$ is normal.
\end{exercise}
\begin{proof}
Let $\mathbb{M}:\mathbb{K}$ be a field extension of degree $2$ and let $a\in\mathbb{M}$ have minimal polynomial $m\in\mathbb{K}[x]$. If $a\in\mathbb{K}$ then everything's fine. Suppose then that $a\not\in\mathbb{K}$. $\mathbb{K}\subseteq\mathbb{K}(a)\subseteq\mathbb{M}$ so $m$ has degree $2$. Write $m$ as $(x-a)(x-b)$ for $b$ in a splitting field of $m$ and suppose that $b\not\in\mathbb{M}$. Then the constant coefficient of $m$ is not in $\mathbb{M}$ and so $m\not\in\mathbb{K}[x]$; a contradiction.
\end{proof}

\begin{exercise}
Show that any automorphism of a field $\mathbb{M}$ is an automorphism  over the prime subfield of $\mathbb{M}$.
\end{exercise}
\begin{proof}
Let $\Phi$ be an automorphism over $\mathbb{M}$ and let $\mathbb{F}$ be the prime subfield. We have $\Phi(1_\mathbb{M})=1_\mathbb{M}$ and so $\Phi(m1_\mathbb{M})=m\Phi(1_\mathbb{M})=m1_\mathbb{M}\forall m\in\mathbb{Z}$. Thus $\Phi(\frac{m1_\mathbb{M}}{n1_\mathbb{M}})=\frac{\Phi(m1_\mathbb{M})}{\Phi(n1_\mathbb{M})}=\frac{m1_\mathbb{M}}{n1_\mathbb{M}}\forall m,n\in\mathbb{Z},n\not\in\langle\text{char}(\mathbb{M}) \rangle$ so $\Phi$ fixes $\mathbb{F}$.
\end{proof}

\begin{exercise}
Show by example that for field extensions $\mathbb{M}:\mathbb{L}:\mathbb{K}$, $\mathbb{M}:\mathbb{L}$ and $\mathbb{L}:\mathbb{K}$ normal $\centernot\implies\mathbb{M}:\mathbb{K}$ normal.
\end{exercise}
\begin{proof}
Let $\mathbb{K}=\mathbb{Q}$, $\mathbb{L}=\mathbb{K}(\sqrt{2})$ and $\mathbb{M}=\mathbb{L}(i\sqrt[4]{2})$. $\mathbb{L}:\mathbb{K}$ is normal since $\mathbb{L}$ is a splitting field of $x^2-2$ over $\mathbb{K}$. $\mathbb{M}:\mathbb{L}$ is normal since $\mathbb{M}$ is a splitting field of $x^2+\sqrt{2}$ over $\mathbb{L}$. $i\sqrt[4]{2}$ is a root of $x^4-2$ which is irreducible over $\mathbb{Q}$ by Eisenstein's criterion. For $\mathbb{M}:\mathbb{K}$ to be normal, we must then have $\sqrt[4]{2}\in\mathbb{M}$. Suppose this is the case. Then $\mathbb{Q}\subseteq\mathbb{Q}(\sqrt{2})\subseteq\mathbb{Q}(\sqrt[4]{2})\subseteq\mathbb{Q}(\sqrt{2},i\sqrt[4]{2})=\mathbb{Q}(i\sqrt[4]{2})$. Then by the tower law, $4=[\mathbb{Q}(\sqrt{2},i\sqrt[4]{2}):\mathbb{Q}]=[\mathbb{Q}(\sqrt{2},i\sqrt[4]{2}):\mathbb{Q}(\sqrt[4]{2})][\mathbb{Q}(\sqrt[4]{2}):\mathbb{Q}(\sqrt{2})][\mathbb{Q}(\sqrt{2}):\mathbb{Q}]=2^3=8$; a contradiction. Thus $\mathbb{M}:\mathbb{K}$ is not normal.
\end{proof}

\begin{exercise}
\begin{enumerate}[label=(\roman*)]
    \item Let $\mathbb{K}$ be a field and let $f$ and $g$ be nonzero polynomials over $\mathbb{K}$. Put $\mathbb{L}=SF_\mathbb{K}(g)$. Show that $SF_\mathbb{L}(f)$ and $SF_\mathbb{K}(fg)$ are isomorphic over $\mathbb{K}$.
    \item Let $f$ and $g$ be nonzero polynomials over $\mathbb{Q}$. Prove that $SF_\mathbb{Q}(fg)$ is the compositum of $SF_\mathbb{Q}(f)$ and $SF_\mathbb{Q}(g)$, where all three splitting fields are viewed as subfields of $\mathbb{C}$.
\end{enumerate}
\end{exercise}

\begin{exercise}
\begin{enumerate}[label=(\roman*)]
    \item $SF_\mathbb{L}(f)$ is the smallest field containing the roots of $f$ and $\mathbb{L}$ - the smallest field containing $\mathbb{K}$ and the roots of $g$ - so $SF_\mathbb{L}(f)$ is the smallest field containing the roots of $f$ and $g$ and $\mathbb{K}$ so is the smallest field containing the roots of $fg$ and $\mathbb{K}$ so is a splitting field of $fg$ over $\mathbb{K}$. Splitting fields are isomorphic so $SF_\mathbb{L}(f)$ and $SF_\mathbb{K}(fg)$ are isomorphic over $\mathbb{K}$.
    \item Let $a_1,...,a_m$ and $b_1,...,b_n$ be the roots of $f$ and $g$ respectively. Then $SF_\mathbb{Q}(fg)=\mathbb{Q}(a_1,...,a_m,b_1,...,b_n)=\mathbb{Q}(a_1,...,a_m)\mathbb{Q}(b_1,...,b_n)=SF_\mathbb{Q}(f)SF_\mathbb{Q}(g)$.
\end{enumerate}
\end{exercise}

\begin{exercise}
Let $0\neq f\in\mathbb{Q}[t]$ with distinct complex roots $\alpha_1,...,\alpha_k$. Prove that $\sum_{i=1}^k\alpha_i^{10}$ is rational.
\end{exercise}
\begin{proof}
Let $\phi\in\text{Gal}(SF_\mathbb{Q}(f)/\mathbb{Q})$. $\phi$ permutes the roots so $\phi(\sum_{i=1}^k\alpha_i^{10})=\sum_{i=1}^k\phi(\alpha_i)^{10}=\sum_{i=1}^k\alpha_i^{10}$. Every $\alpha\in SF_\mathbb{Q}(f)\backslash\mathbb{Q}$ has an automorphism in $\text{Gal}(SF_\mathbb{Q}(f)/\mathbb{Q})$ which doesn't fix it so $\sum_{i=1}^k\alpha_i^{10}$ must be rational.
\end{proof}

\begin{exercise}
Let $\mathbb{L}:\mathbb{K}$ be an algebraic extension. Prove that $\mathbb{L}:\mathbb{K}$ is normal if and only if it has the following property: for every extension $\mathbb{M}:\mathbb{L}$, the field $\mathbb{L}$ is a union of conjugacy classes in $\mathbb{M}$ over $\mathbb{K}$.

(Conjugacy over $\mathbb{K}$ defined an equivalence relation of $\mathbb{M}$, and a 'conjugacy class in $\mathbb{M}$ over $\mathbb{K}$' means an equivalence class of this equivalence relation.)
\end{exercise}
\begin{proof}
$(\implies)$ Let $\mathbb{M}:\mathbb{L}$ be an extension. Let $a\in\mathbb{L}$ and let $b\in\mathbb{M}$ be conjugate to $a$ over $\mathbb{K}$. Then $p(a)=0\iff p(b)=0\forall p\in\mathbb{K}[x]$. Let $f\in\mathbb{K}[x]$ be an irreducible polynomial which has $a$ as a root. Then $b$ is also a root of $f$ and so $b\in\mathbb{L}$ since $\mathbb{L}:\mathbb{K}$ is normal. Thus if $\mathbb{L}$ contains $a$, it contains the whole conjugacy class of $a$ in $\mathbb{M}$ and so $\mathbb{L}$ is a union of conjugacy classes in $\mathbb{M}$ over $\mathbb{K}$.

$(\impliedby)$ Let $a\in\mathbb{L}$ and let $f\in\mathbb{K}[x]$ be an irreducible polynomial with $a$ as a root. Let $b\in\mathbb{M}$ be another root of $f$. Then $\text{Ker}(\text{ev}_a)=\langle f\rangle=\text{Ker}(\text{ev}_b)$ and so $a$ and $b$ are conjugate so are in the same conjugacy class. $\mathbb{L}$ is a union of conjugacy classes in $\mathbb{M}$ so $b\in\mathbb{L}$. Thus $\mathbb{L}:\mathbb{K}$ is normal.
\end{proof}


\begin{exercise}
Let $f$ be an irreducible cubic over $\mathbb{Q}$. Write $\alpha_1,\alpha_2,\alpha_3$ for the complex roots of $f$, and put\[\delta=(\alpha_1-\alpha_2)(\alpha_1-\alpha_3)(\alpha_2-\alpha_3).\]
\begin{enumerate}[label=(\roman*)]
    \item Show that $\text{Gal}_\mathbb{Q}(f)$ is isomorphic to $A_3$ or $S_3$.
    \item Show that $\delta\neq 0$.
    \item Show that $\theta(\delta)=\pm\delta$ for all $\theta\in\text{Gal}_\mathbb{Q}(f)$.
    \item Show that\[G\cong\begin{cases}
       A_3 &\quad\text{if }\delta\in\mathbb{Q}, \\
       S_3 &\quad\text{otherwise.} \\ 
     \end{cases}\]
     \item Define\[\Delta=\delta^2=(\alpha_1-\alpha_2)^2(\alpha_1-\alpha_3)^2(\alpha_2-\alpha_3)^2.\]It is tedious but straightforward to check that if we write\[B=-(\alpha_1+\alpha_2+\alpha_3),C=\alpha_1\alpha_2+\alpha_1\alpha_3+\alpha_2\alpha_3,D=-\alpha_1\alpha_2\alpha_3\]then\[\Delta=-27D^2+18BCD-4C^3-4B^3D+B^2C^2.\]Which result from Chapter $8$ implies that $\Delta\in\mathbb{Q}$, with zero calculation?
     \item Deduce that if we write $f(t)$ as $t^3+bt^2+ct+d$ then\[\text{Gal}_\mathbb{Q}(f)\cong\begin{cases}
       A_3 &\quad\text{if }\sqrt{-27d^2+18bcd-4c^3-4b^3d+b^2c^2}\in\mathbb{Q}, \\
       S_3 &\quad\text{otherwise.} \\ 
     \end{cases}\]
     \item Find the Galois group of $t^3-3t-1$.
\end{enumerate}
\end{exercise}

\begin{proof}
\begin{enumerate}[label=(\roman*)]
    \item $f$ is irreducible so $[\mathbb{Q}(\alpha_i):\mathbb{Q}]=3\forall i$. Thus $3$ divides $[\mathbb{Q}(\alpha_1,\alpha_2,\alpha_3):\mathbb{Q}]=|\text{Gal}_\mathbb{Q}(f)|$. $\text{Gal}_\mathbb{Q}(f)$ is isomorphic to a subgroup of $S_3$ (Char$(\mathbb{Q})=0$ so $f$ is separable) and the only subgroups of $S_3$ with orders divisible by $3$ are $A_3$ and $S_3$.
    \item The roots are all distinct so $\delta\neq0$.
    \item $\theta$ permutes $\alpha_1,\alpha_2,\alpha_3$ so $\theta(\delta)=\pm\delta$.
    \item If $\delta\in\mathbb{Q}$ then every $\theta\in\text{Gal}_\mathbb{Q}(f)$ fixes $\delta$. The automorphism given by swapping $\alpha_1$ and $\alpha_2$ switches the sign of $\delta$ so $G$ cannot be isomorphic to $S_3$ if $\delta\in\mathbb{Q}$ so must be isomorphic to $A_3$. If $\delta\not\in\mathbb{Q}$ then there must exist a $\theta\in\text{Gal}_\mathbb{Q}(f)$ such that $\theta(\delta)=-\delta$. Every $\theta$ corresponding to an element of $A_3$ fixes $\delta$ so there must be a $\theta$ in the Galois group which doesn't correspond to an element of $A_3$ and so $G\cong S_3$ if $\delta\not\in\mathbb{Q}$.
    \item $\Delta$ is fixed by every automorphism in the Galois group so $\Delta\in\mathbb{Q}$.
    \item It's the same condition as asking if $\delta$ is rational.
    \item $\sqrt{-27-4(-3)^3}=9$ so the Galois group is $A_3$.
\end{enumerate}
\end{proof}

\begin{exercise}
Work through the details of the Galois correspondence for $t^4-2t^2+9\in\mathbb{Q}[t]$.
\end{exercise}
\begin{proof}
Reduce $f$ mod $3$ to become $\overline{f}=t^4++t^2$
\end{proof}

\begin{exercise}
Let $p$ be a prime. Prove that $\text{Gal}_\mathbb{Q}(t^p-1)\cong C_{p-1}$.
\end{exercise}

\begin{exercise}
\begin{enumerate}[label=(\roman*)]
    \item Show that when $p$ is prime, $\Phi_p(t)=t^{p-1}+...+t+1$.
    \item Calculate $\Phi_n$ for $n=1,...,7$.
    \item By considering $\theta_*\Phi_n$ for $\theta\in\text{Gal}_\mathbb{Q}(t^n-1)$, prove that $\Phi_n\in\mathbb{Q}[t]$.
    \item Show that $\prod_{d|n}\Phi_d(t)=t^n-1$.
    \item use Gauss's lemma on primitive polynomials to show that whenever $f,g\in\mathbb{Q}[t]$ are monic polynomials such that $fg\in\mathbb{Z}[t]$, then $f,g\in\mathbb{Z}[t]$.
    \item Put together the previous parts to conclude that $\Phi_n\in\mathbb{Z}[t]$.
\end{enumerate}
\end{exercise}


\end{document}
