\documentclass{article}
\usepackage{graphicx} % Required for inserting images
\usepackage[english]{babel}
\usepackage{amssymb}
\usepackage{amsthm}
\usepackage{amsmath}
\usepackage{amsfonts}
\usepackage{tikz}
\usetikzlibrary{matrix}
\usepackage[english]{babel}
\usepackage{mathtools}
\usepackage[a4paper, total={6in, 8in}]{geometry}
\usepackage{cite}
\title{Galois theory solutions}
\author{saxon.supple }
\date{July 2024}


\newtheorem{theorem}{Theorem}[section]
\newtheorem{definition}[theorem]{Definition}
\newtheorem{lemma}[theorem]{Lemma}
\newtheorem{proposition}[theorem]{Proposition}
\newtheorem{corollary}[theorem]{Corollary}
\newtheorem{example}[theorem]{Example}
\newtheorem{remark}[theorem]{Remark}
\newtheorem{exercise}[theorem]{Exercise}

\begin{document}

\maketitle

\section{Groups}

\subsection{Groups}
\begin{exercise}
Suppose That $G$ is a group. Show that the identity element $e$ is unique, and that for each $g\in G$ the inverse element $g^{-1}$ is also unique.
\end{exercise}
\begin{proof}
Let $e_1$ and $e_2$ be identity elements of $G$. Then $e_1=e_1e_2=e_2$ so identity elements are unique.\\
Let $g_1^{-1}$ and $g_2^{-1}$ be inverses of $g$. Then $e=g_1^{-1}g=g_2^{-1}g\implies g_1^{-1}=g_2^{-1}$ so inverses are unique.
\end{proof}

\begin{exercise}
Show that if $H$ is a subgroup of $\mathbb{Z}$, then $H$ is cyclic.
\end{exercise}
\begin{proof}
There exists a greatest common divisor of the elements of $H$. Call it $d$. $d\in H$ by Bezout's lemma. Furthermore, every element of $H$ is a multiple of $d$. Thus $d$ is a generator of $H$ so $H$ is cyclic.
\end{proof}

\begin{exercise}
Show that a subgroup $F$ of a cyclic group $G$ is cyclic.
\end{exercise}
\begin{proof}
Let $g$ be a generator of $G$. Then every element $f\in F$ is of the form $g^n$ for some $n\in \mathbb{Z}$. Let $H$ be the set of all integers $n$ such that $g^n\in F$. $e=g^0\in F$ so $0\in H$. $g^n\in F\implies g^{-n}\in F$ so $H$ is closed under inverses. $g^n,g^m\in F\implies g^ng^m=g^{n+m}\in F$ so $H$ is closed under composition. Thus $H$ is a subgroup of $\mathbb{Z}$ so is cyclic. Let $d$ be a generator of $H$. every element of $F$ is then of the form ${(g^d)}^k$ for some integer $k$. Thus $F$ is cyclic.
\end{proof}

\begin{exercise}
Suppose that $A$, $A_1$ and $A_2$ are subsets of a group $G$, and that $A_1\subseteq A_2$.
Show that\\
\begin{align*}
(i&) Z(A)\text{ is a subgroup of }G\\
(ii&) Z(A_2)\subseteq Z(A_1)\\
(iii&) A\subseteq Z(Z(A))\\
(iv&) Z(A)=Z(Z(Z(A)))\\
(v&) A\subseteq Z(A)\text{ if and only if }A\text{ is abelian.}
\end{align*}
\end{exercise}

\begin{proof}
$(i)$\\
$e^{-1}g^{-1}eg=e\forall g\in G$ so $e\in Z(A)$.\\
Let $a\in Z(A)$. Let $g\in A$. Then 
\begin{align*}   
a^{-1}g^{-1}ag&=e\\\implies ag&=ga\\\implies g^{-1}ag&=a\\\implies g^{-1}a&=ag^{-1}\\\implies ag^{-1}a^{-1}&=g^{-1}\\\implies ag^{-1}a^{-1}g&=e
\end{align*}
so $a^{-1}\in Z(A)$.

Let $a,b\in Z(A),g\in A.$ $(ab)^{-1}g^{-1}(ab)g=b^{-1}a^{-1}g^{-1}abg=b^{-1}g^{-1}bg=e$ so $ab\in Z(A)$.
Thus $Z(A)$ is a subgroup of $G$.

$(ii)$\\
Let $a\in Z(A_2)$. Let $g\in A_1$. Then $g\in A_2$ so $[a,g]=e$. Thus $a\in Z(A_1)$.

$(iii)$\\
Let $a\in A$. We want $[a,g]=e\forall g\in Z(A)$. If $g\in Z(A)$, then $[g,a]=e\implies [a,g]=e$.

$(iv)$\\
$A\subseteq Z(Z(A))$ so $Z(Z(Z(A)))\subseteq Z(A)$. Furthermore, $Z(A)\subseteq Z(Z(Z(A)))$ so $Z(A)=Z(Z(Z(A)))$.

$(v)$\\
Let $a,b\in A$. Then $a\in Z(A)$ so $[a,b]=0$ so $A$ is abelian.

Now let $A$ be abelian. Then every element of $A$ commutes with every element of $A$ so $A\subseteq Z(A)$.
\end{proof}

\begin{exercise}
Show that if $\phi$ is a homomorphism of a group $G$ into a group $H$ then its kernel is a normal subgroup of $G$.
\end{exercise}
\begin{proof}
let $a\in\text{ker }\phi$ so that $\phi(a)=e$. Let $g\in G$. Then $\phi(g^{-1}ag)=\phi(g^{-1})\phi(a)\phi(g)=\phi(g)^{-1}\phi(g)=e$ so $a^g\in \text{ker }\phi$. Thus $\text{ker }\phi$ is normal.
\end{proof}

\begin{exercise}
Suppose that $A$ is a non-empty subset of a group $G$ and that $g\in G$. let $\psi_A(g)=A^g$. Show that $\psi_A$ maps $G$ onto conj$(A)$, and $\psi_A(g')=\psi_A(g)$ if and only if $g'\in N(A)g$. Thus if $G$ is a finite group then $|\text{conj}(A)|=|G/N(A)|$, so that $|\text{conj}(A)|\cdot|N(A)|=|G|$.
\end{exercise}
\begin{proof}
$\psi_A(g)=A^g$ so $\psi_A(g)$ is conjugate to $A$. $\psi_A$ is also clearly surjective so maps $G$ onto conj$(A)$.

$\psi_A(g')=\psi_A(g)\iff A^{g'}=A^g$.

Let $g'\in N(A)g$. Then $g'=hg$ for some $h\in G$ such that $a^h\in A\forall a\in A$. Thus given $a\in A$, $a^{g'}=(hg)^{-1}ahg=g^{-1}h^{-1}ahg=g^{-1}a^hg\in A^g$. Hence $A^{g'}\subseteq A^g$. Furthermore, $g=h^{-1}g'$ so $A^g\subseteq A^{g'}$. Thus $A^{g'}=A^g$.

Now let $A^{g'}=A^g$. We want $a^{g'g^{-1}}\in A\forall A$. Let $a\in A$. $a^{g'g^{-1}}=gg'^{-1}ag'g^{-1}=ga^{g'}g^{-1}=(b^g)^{g^{-1}}$ for some $b\in A$. Thus $a^{g'g^{-1}}\in A$.

Let $G$ be finite. We have established a correspondence between elements of conj$(A)$ and cosets of $N(A)$ so $|\text{conj}(A)|=|G/N(A)|$. $|\text{conj}(A)|\cdot|N(A)|=|G|$ since $|G/N(A)|=|G|/|N(A)|$. 
\end{proof}
\end{document}
